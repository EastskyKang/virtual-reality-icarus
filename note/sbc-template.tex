\documentclass[12pt]{article}

\usepackage{sbc-template}

\usepackage{graphicx,url}

\usepackage[english]{babel}   
%\usepackage[latin1]{inputenc}  
\usepackage[utf8]{inputenc}  
% UTF-8 encoding is recommended by ShareLaTex
\usepackage{verbatim}
\usepackage{listings}
\usepackage{xcolor}
\usepackage{graphicx}
\usepackage{listings}
\usepackage[]{algorithm2e}
\usepackage{color}
\usepackage{amsmath}
\usepackage{amsfonts}
\usepackage{multirow}
\RestyleAlgo{ruled}
\usepackage{fancybox}
\usepackage{siunitx}


\newcommand\norm[1]{\left\lVert#1\right\rVert}
\definecolor{verde}{rgb}{0,0.5,0}
\DeclareMathAlphabet{\mathpzc}{OT1}{pzc}{m}{it}
%para customizar o código (ver https://en.wikibooks.org/wiki/LaTeX/Source_Code_Listings)
\lstset{language=C, %defina a linguagem usada no trabalho
              belowcaptionskip=1\baselineskip,
                breaklines=true,
                frame=false,
                xleftmargin=\parindent,
                showstringspaces=false,
                basicstyle=\footnotesize\ttfamily,
                keywordstyle=\bfseries\color{green!40!black},
                commentstyle=\itshape\color{purple!40!black},
                identifierstyle=\color{blue},
                stringstyle=\color{orange},
                numbers=left,
            }

\sloppy

\title{Virtual Reality 1}

\author{Dongho Kang, Jaeyoung Lim, Soomin Lee, Jaeryeong Choi}


\address{Choseon
}

\begin{document} 

\maketitle

\section{Chapter 1: Introduction into Virtual Reality}

\subsection{Formation of Virtual Reality}

\subsubsection*{The concept of VR}

\begin{itemize}
	\item capture things and thoughts which elude human 
	\begin{itemize}
		\item painting and stone sculpture (e.g. painting on the walls of cave) 
 		\item complex physical models (e.g. Newton's Law)
	\end{itemize} 
	\item capture (copy) the reality \& portray the reality for the purpose of a better understanding
\end{itemize}

\subsubsection*{VR before computer}

\begin{itemize}
	\item The first projector : Laternae Magicae (17c) \textbf{Figure 1.2, page 1-2}
		\begin{itemize}
			\item optical system. a picture painted on a glass sheet and illuminated from behind with a candle
			\item monoscopic 
		\end{itemize}
	\item Mirror-based stereoscope (C. Wheatstone, 1832) \textbf{Figure 1.3, page 1-2}
		\begin{itemize}
			\item user can gain the impression of depth 
			\item very limited field of view
		\end{itemize}
	\item Panorama (R. Baker, 1787) \textbf{Figure 1.4, page 1-3}
		\begin{itemize}
			\item building (built at Leicester Square in London)  
			\item wider image than stereoscope: cylindrical images from height 10 ~ 14m, circumference 140m
			\item still image (no move)
		\end{itemize}
	\item Kaiserpanorama (19c) \textbf{Figure 1.5, page 1-3}
		\begin{itemize}
			\item moving panorama image
		\end{itemize}
	\item Head-Mounted-Display (Heilig \& Sutherland, 1960) \textbf{Figure 1.6, page 1-4}
		\begin{itemize}
			\item advances in the human-machine-interfacee, electronics and graphics hardware
			\item the rise of VR in the entertainment industries
		\end{itemize}
	\item Sensorama Simulator (M. Heilig, 1962) \textbf{Figure 1.7, page 1-4}
		\begin{itemize}
			\item one person arcade game (control of bikes and cars)
			\item stereovision / stereo sound / wind / realized scents 
			\item was not successful in the market
		\end{itemize}
	\item Waller Gunnery Trainer  \textbf{Figure 1.8, page 1-5}
		\begin{itemize}
			\item 5 cameras and 5 projectors
			\item used to train soldiers for the air force
		\end{itemize}
	\item Cinerama \textbf{Figure 1.9$\sim$11, page 1-6$\sim$7}
		\begin{itemize}
			\item 3 special cameras and 3 projectors (146\si{\degree} wide 55\si{\degree} hight) but some problems...
				\begin{itemize}
					\item stitch problem
					\item geometric distortion
					\item color temperature differences
					\item sync by hand (Theatre engineers)
					\item odd orientation of the actors (no eye contact)
				\end{itemize}
			\item cylindrical shape screen
		\end{itemize}
\end{itemize}

\subsection{The development of the computer}


\subsection{Definitions}


\section{Chapter 2}

\section{Chapter 3}

\section{Chapter 4}

\end{document}
